\documentclass[a4paper]{article}

%% Language and font encodings
\usepackage[english]{babel}
\usepackage[utf8x]{inputenc}
\usepackage[T1]{fontenc}

%% Sets page size and margins
\usepackage[a4paper,top=3cm,bottom=2cm,left=3cm,right=3cm,marginparwidth=1.75cm]{geometry}

%% Useful packages
\usepackage{amsmath}
\usepackage{graphicx}
\usepackage[colorinlistoftodos]{todonotes}
\usepackage[colorlinks=true, allcolors=blue]{hyperref}

\title{Alumni Portal}
\author{Anant Mishra, Bharat Kumar }
\date {January 27, 2018}

\begin{document}
\maketitle

\begin{abstract}
Alumni Portal is a web portal where alumni of particular university can register themselves and build a social and professional network with students and facility of their alma-mater.
\end{abstract}

\section{Objective}

The aim of this project is to build a system that will be able to manage alumni data of the university and provide easy access for the same. Where new students will be initially given a student login ID by that they can access their accounts, alumni can do the same. 
    
    The portal will automatically list all college students as alumni on their graduation. Users will be also prompted to update their social network details such as Facebook, LinkedIn and Twitter handles. Users of the portal can also choose to automatically share new updates in work status from their LinkedIn profile, they will also be able to share their Facebook and Twitter updates. 
    
	Alumni will also be able to provide public posts on the portal about possible job opportunities, new technologies \& business possibilities or other university related news. Since it is unlikely that alumni will check the portal frequently so the portal will be able collate all public posts and create a newsletter that can be emailed to all alumni.
    
	The portal will also have some privacy features where users will be able to determine what information they want to share and also to whom they want to share it with. For example : users can choose to share their Facebook profile name and mobile number with alumni who graduated in the same year as them. 
    
    They portal will also have a chat feature which will enable alumni to chat without revealing their mobile number or personal e–mail IDs. 

\section{Motivation}

Being a 3rd year student of the university we have realize that it is very hard to communicate with alumnus. Alumni of a college generally stay in touch with their immediate friends but find it hard to stay connected with other college mates. 

The contact between alumni can be used to forge business connections and to gain references or insight in a new field. It can also be used in getting help in projects or for placements.

This part gave us a motivation towards building and coding a system where freshman of the university to the senior most student can easily [be] get connected with alumnus.

\section{Modules}
\subsection{Admin}
The admin will be responsible for creating new login IDs for incoming students, admin will also have to browse the site to ensure no objectionable content is posted. The admin will also be notified about any complaints from users.
\subsection{Student}
The student module can be used to browse through the site and access alumni information. The students will be able to chat via the portal with the alumni, if the alumni wishes to share e–mail and mobile number this can be done through the chat. The student will have to seek admin approval before posting anything on the site.
\subsection{Alumni}
An alumnus of the college will be able to access other alumni information and also will be able to view all their contact information (unless it is made private). An alumnus can be able to post any information they deem relevant on the site.

\section{Software Requirements}

\begin{itemize}
\item Operating system (e.g. Windows, Linux),
\item IDE (PyCharm),
\end{itemize}

\section{Technology Used}

\begin{itemize}
\item Python
\item Django
\item Other web technologies such as : HTML5, CSS3, Bootstrap, AJAX, JavaScript etc. 
\end{itemize}

\section{Hardware Requirements}

\begin{itemize}
\item Hard Disk – 20 GB
\item RAM – 1 GB
\item Processor – Dual Core or Above
\end{itemize}

\bigskip
\baselinestretch Project proposal by:
\begin{itemize}
\item Anant Mishra, 11152544, B.Tech CSE
\item Bharat Kumar, 11152607, B.Tech CSE
\end{itemize}

\end{document}